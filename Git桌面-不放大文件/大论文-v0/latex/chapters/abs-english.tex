%英文摘要,自行编辑内容




\chapter{ABSTRACT}
\xiaosi

With the increasing demand for data privacy protection and the widespread existence of distributed data environments, Federated Learning (FL) has rapidly developed as a distributed machine learning paradigm that safeguards data privacy. However, in practical applications, federated learning faces challenges such as scarcity of labeled data, heterogeneous data distributions, and insufficient sample alignment among participants. Semi-Supervised Learning (SSL), with its ability to effectively leverage a small amount of labeled data and a large amount of unlabeled data to improve model performance, has become a key approach to addressing these issues. This thesis focuses on the study of sample generation methods based on federated semi-supervised learning, aiming to design innovative algorithms and frameworks that fully utilize unlabeled data and non-aligned samples while protecting data privacy, thereby enhancing the generalization capability and practical value of federated learning models.

Firstly, this thesis addresses the issue of missing unlabeled data in multi-party federated learning by proposing a federated semi-supervised learning method called VFPU, which integrates Positive-Unlabeled (PU) learning. Based on an in-depth analysis of the unlabeled data deficiency problem (UDD-PU), this method constructs an innovative federated collaboration framework. Through encrypted sample alignment, dynamic pseudo-label generation, and ensemble learning strategies, it effectively leverages decentralized unlabeled data from multiple parties to train the model. While ensuring privacy protection, the VFPU method significantly improves the precision and recall of federated recommendation tasks. Experimental results demonstrate that its performance is comparable to centralized methods, successfully achieving an effective balance between model performance and data privacy protection.

Second, to tackle the limitations of Vertical Federated Learning (VFL), where the number of aligned samples is limited and non-aligned samples are underutilized, this thesis proposes a sample generation method based on semi-supervised learning, VFPU-M-Syn. This method generates high-quality pseudo-labels and synthetic data through three strategies: cross-party feature correlation analysis, semi-supervised prediction, and generative model synthesis. Specifically, VFPU-M-Syn calculates cross-party feature correlations to identify dependencies between participant data, uses vertical federated semi-supervised learning to predict pseudo-labels for unlabeled samples, and introduces Generative Adversarial Networks (GANs) to synthesize data for features with low correlation.

The research outcomes of this thesis not only enrich the theoretical framework of federated semi-supervised learning but also provide feasible solutions for distributed learning scenarios requiring privacy protection, such as healthcare, finance, and intelligent manufacturing. Future work will further optimize the algorithm’s adaptability to Non-Independent and Identically Distributed (Non-IID) data, explore dynamic threshold adjustment mechanisms, and extend its applications to cross-modal data fusion, offering technical support for the efficient utilization and secure sharing of data elements.
\\


\noindent\textbf{Keywords:} 
\begin{minipage}[t]{0.85\linewidth}
	Federated Learning; Semi-Supervised Learning; Sample Generation; Privacy Protection; Vertical Federated Learning
\end{minipage}

\clearpage