\chapter{总结与展望}

学位论文应有结论,可以从论文的主要工作、创新点和后续的研究工作等方面进行总结。

\section{主要工作与创新点}

学位论文的结论是最终的、总体的结论,不是正文中各段的小结的简单重复。结论应该观点明确、严谨、完整、准确、精炼。文字必须简明扼要。可以从论文的主要工作、创新点和后续的研究工作等方面总结。

如果不可能导出应有的结论,也可以没有结论而进行必要的讨论。

可以在结论或讨论中提出建议、研究设想、仪器设备改进意见、尚待解决的问题等。不要简单重复罗列实验结果,要认真阐明本人在科研工作中创造性的成果和新见解,在本领域中的地位和作用,新见解的意义。对存在的问题和不足应作出客观的叙述。应严格区分自己的成果与他人(特别是导师的)科研成果的界限。

一般应按四级标题的方式给出,根据需要设置数量。如本文主要工作和创新点如下:

1)阐述第一个创新工作。不要把阅读文献当成创新工作。

2)阐述第二个创新工作。

3)阐述第三个创新工作。

4)阐述第四个创新工作。

特别提醒,不应简单和中文摘要内容相互拷贝。同一段文字或句子在本文中原则上只出现一次。

\section{后续研究工作展望}

针对工作不足或问题,说明更下一步深入的研究。如内容较多,也应用四级标题方式列出。