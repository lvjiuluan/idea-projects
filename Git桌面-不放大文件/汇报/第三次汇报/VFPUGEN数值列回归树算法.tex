\documentclass{article}
\usepackage{fontspec}
\setmainfont[Mapping=tex-text]{SimSun}
\title{VFPUGEN numerical column regression tree algorithm}
\author{jiuluan lv}
\date{July 2024}
\usepackage{algorithm}
\usepackage{algorithmicx}
\usepackage{algpseudocode}
\usepackage{amsmath}
\usepackage[UTF8]{ctex}
\usepackage{hyperref}

\renewcommand{\algorithmicrequire}{\textbf{输入:}}
\renewcommand{\algorithmicensure}{\textbf{输出:}}
\renewcommand{\theequation}{(\arabic{equation})}
\numberwithin{equation}{algorithm}


\begin{document}
	
	\maketitle
	
	\begin{algorithm}
		\caption{数值列采用回归树处理算法}
		\begin{algorithmic}[0]
			\Require 训练数据集$D$
			\Ensure 回归树$f(x)$
			\State 在训练数据集所在的输入空间中,递归地将每个区域划分为两个子区域并决定每个子区域上的输出值,构建二叉决策树:
			\State (1)选择最优切分变量$j$与切分点$s$,求解
			\begin{equation}
				\min_{j,s} \left[ \min_{c_1} \sum_{x_i \in R_1(j,s)} (y_i - c_1)^2 + \min_{c_2} \sum_{x_i \in R_2(j,s)} (y_i - c_2)^2 \right]
			\end{equation}
			遍历变量$j$,对固定的切分变量$j$扫描切分点$s$,选择使式(1.1)达到最小值的对$(j,s)$
			\State (2)用选定的对$(j,s)$划分区域并决定相应的输出值:
			\begin{align}
				R_1(j,s) &= \{ x|x^{(j)} \le s \}, R_2(j,s) = \{ x|x^{(j)} > s \} \\
				\hat{c}_m &= \frac{1}{N_m}\sum_{x_i \in R_m(j,s)} y_i, x \in R_m, m = 1,2
			\end{align}
			\State (3)继续对两个子区域调用步骤(1),(2),直至满足停止条件
			\State (4)将输入空间划分为$M$个区域$R_1,R_2,...,R_M$,生成决策树:
			\begin{equation}
				f(x) = \sum_{m = 1}^M \hat{c}_mI(x \in R_m)
			\end{equation}
		\end{algorithmic}
	\end{algorithm}
	
\end{document}
