\chapter{其他格式要求}

学位论文中对其他的格式要求还要页面要求、页眉页脚要求、正文的层次安排、打印要求和论文查非要求等。

\section{页面要求}

学位论文用A4(210×297mm)纸,采用双面打印,装订成品尺寸:207×291mm。

论文外部封面采用学校当年统一印刷提供的封面。此论文模板封面为内封,应是打印论文时的内页首页。评审、答辩等中间环节提供的纸质论文可不用学校统一外封面正式装订,而是用本模板封面作为临时封面。

\section{页眉}

从目录页开始到论文最后一页,均需设置页眉。页眉内容:偶数页居中对齐为“重庆邮电大学本科毕业设计(论文)”,奇数页居中对齐是各章章名;字体采用宋体5号。页眉之下有一条下划线。封面、摘要没有页眉,也没有边框。

\section{页脚}

页脚放置页码,页码在版芯下边线之下隔行居中放置;摘要、目录、图录、表录、注释表等文前部分的页码用罗马数字单独编排,正文以后,从引言开始的页码用阿拉伯数字连续编排。

\section{打印要求}
\subsection{页面设置}

学位论文页边距按以下标准设置:上边距(天头)3厘米,下边距(地脚)2.5厘米,左右边距2.5厘米,装订线靠左1厘米,页眉顶端距离1.6厘米,页脚底端距离1.5厘米。无网格。

\subsection{字体}

论文正文的中文字体用宋体;英文字体则要求为Times New Roman。正文中的文字部分要求两端对齐。

\subsection{字号}

1)目录题目(目录、图录、表录、注释)——是一级标题,中文黑体、英文Times New Roman, 3号字居中,段前17磅,段后16.5磅,1.5倍行距;

2)章标题(第x章)——是一级标题,中文黑体、英文Times New Roman, 3号字居中,段前17磅,段后16.5磅,1.5倍行距;

3)节标题(x.x)——是二级标题,中文黑体、英文Times New Roman小3号字顶格居左,段前13磅,段后13磅,1.5倍行距,节名和文字间空1个字符,不空行;

4)条标题(x.x.x)——是三级标题,中文黑体、英文Times New Roman4号字顶格居左,段前13磅,段后13磅,1.5倍行距,条名和文字间空1个字符,不空行;

5)正文——中文宋体、英文Times New Roman小4号,首行缩进2字符,1.5倍行距;

6)正文后的题目(参考文献、致谢、攻读xx期间发表的论文)——是一级标题,中文黑体、英文Times New Roman, 3号字居中,段前17磅,段后16.5磅,1.5倍行距。


\section{论文查重要求}

为进一步提高我校本科毕业(设计)论文教学质量,加强规范管理,科学引用文献资料,树立良好学风,防止本科毕业(设计)论文抄袭、拷贝、篡改已有科研成果等学术不端现象的发生,学校决定使用“大学生论文抄袭检测系统”对毕业生的毕业(设计)论文进行检测。

\textbf{1)检测范围}

毕业(设计)论文均可通过“维普论文检测系统”进行检测,各学院可根据本学院各专业的实际情况设定本科毕业(设计)论文进行检测的抽查规则,但抽检的论文占总人数的比例不得低于45%。
维普论文检测系统网址:http://vpcs.cqvip.com/login.aspx?sysid=2

\textbf{2)检测标准}

检测结果中,毕业(设计)论文文字复制比(即毕业(设计)论文的某一章节与比对文献比较后,重合文字部分在该章节中所占的比例)在30%(含30%)以内的视为合格,文字复制比超过30%的视为不合格。参评校级优秀毕业(设计)论文必须经过检测并且复制比应控制在20%以内。

\textbf{3)检测不合格处理}

检测不合格的毕业(设计)论文,给予一次修改机会,经再次检测合格后方可申请答辩与成绩评定;复检仍不合格的延期答辩。

\section{论文非学术性错误}
论文中的非学术型错误主要包含错别字、格式错误、图表错误、参考文献格式错误、序号列表错误、排版问题等被普遍认为的非学术型的低级错误。论文中存在大量低级错误,说明论文写作质量差,影响读者对其学术水平的判定,同时也反映了该论文作者缺乏认真、严谨、负责的科学态度和素养,没有达到合格人才的培养目的。学位论文写作中非学术性错误的主要表现见附录A。

\section{本章小结}
介绍了学位论文的其他格式要求,以及学校关于检查论文中非学术性错误(俗称低级错误)的要求。







