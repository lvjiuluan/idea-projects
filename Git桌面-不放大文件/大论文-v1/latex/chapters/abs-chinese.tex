%中文摘要,自行编辑内容



\chapter{摘\quad 要}
\xiaosi

随着数据隐私保护需求的日益增加以及分布式数据环境的广泛存在,联邦学习(Federated Learning, FL)作为一种保护数据隐私的分布式机器学习范式得到了快速发展。然而,在实际应用中,联邦学习面临标记数据稀缺、数据分布异构以及参与方样本对齐不足等挑战。半监督学习(Semi-Supervised Learning, SSL)因其能够有效利用少量标记数据和大量未标记数据提升模型性能,成为解决上述问题的重要手段。本文聚焦于基于联邦半监督学习的样本生成方法研究,旨在通过设计创新的算法和框架,在保护数据隐私的前提下,充分利用未标记数据和非对齐样本,提升联邦学习模型的泛化能力和应用价值。  

(1)针对多方联邦学习中未标记数据缺失的问题,提出了一种结合正样本-未标记样本学习(Positive-Unlabeled, PU)的联邦半监督学习方法VFPU。该方法基于对未标记数据缺失问题(UDD-PU)的深入分析,构建了一种创新的联邦协作框架,通过加密样本对齐、动态伪标签生成和集成学习策略,有效利用多方分散的未标记数据来训练模型。在隐私保护的基础上,VFPU方法显著提升了联邦推荐任务的精确率与召回率,实验证明其性能与集中式方法相当,成功地在模型性能和数据隐私保护之间实现了有效平衡。 

(2)针对纵向联邦学习(Vertical Federated Learning, VFL)中对齐样本数量有限、非对齐样本未被充分利用的局限性,本文提出了基于半监督学习的样本生成方法FedPSG-PUM。该方法通过跨方特征相关性分析、半监督预测和生成模型合成三种策略,生成高质量的伪标签和合成数据。具体而言,FedPSG-PUM计算跨方特征相关性以识别参与方数据间的依赖关系,利用纵向联邦半监督学习预测未标记样本的伪标签,并针对低相关性特征引入生成对抗网络(GAN)合成数据。 
  
\noindent\songti\textbf{关键词:}联邦学习;半监督学习;样本生成;隐私保护;纵向联邦学习

\clearpage