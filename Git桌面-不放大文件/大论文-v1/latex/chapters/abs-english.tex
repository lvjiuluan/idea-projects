%英文摘要,自行编辑内容




\chapter{ABSTRACT}
\xiaosi

With the increasing demand for data privacy protection and the widespread presence of distributed data environments, federated learning (FL), as a distributed machine learning method protecting data privacy, has been rapidly developed. However, in practical applications, federated learning faces several challenges, including scarcity of labeled data, heterogeneous data distribution, and insufficient aligned samples among participating parties. Semi-supervised learning (SSL), which effectively utilizes limited labeled data and abundant unlabeled data to enhance model performance, has emerged as a significant approach to addressing these challenges. This paper focuses on federated semi-supervised learning methods and their applications in sample generation, aiming to improve the generalization ability and application value of federated learning models by fully utilizing unlabeled data and non-aligned samples under the premise of preserving data privacy. The primary research contents of this paper are as follows:

(1) To address the issue of unlabeled data absence in multi-party federated learning, a federated semi-supervised learning approach named VFPU is proposed, combining Positive-Unlabeled (PU) learning. VFPU repeatedly performs random sampling from multi-party unlabeled data, temporarily treating the sampled data as negative samples, thus forming multiple balanced training datasets (in terms of positive-negative ratio) and multiple testing datasets composed of unsampled data. Base learners are iteratively trained within a vertical federated learning framework on each training dataset. Subsequently, trained base classifiers generate predictive scores for each sample in the testing datasets. The probability of each unlabeled sample being positive is calculated based on its frequency of occurrence in the testing datasets and the sum of its scores. Samples with the highest probability are identified as reliable positive samples, then added to the positive set and removed from the unlabeled data. This iterative process of sampling, training, and selecting positives continues repeatedly. Experimental results demonstrate that the proposed method achieves comparable performance to similar non-federated approaches and outperforms other federated semi-supervised methods.

(2) To address the limited quantity of aligned samples in vertical federated learning, this paper proposes a sample generation method combining semi-supervised learning with data generation and imputation techniques. This method effectively improves data utilization and model performance by integrating federated semi-supervised learning with generative models. Specifically, the method consists of three core processes: first, cross-party feature correlations are calculated via privacy-preserving Spearman rank correlation analysis to construct a ranking system of feature association strength; second, for highly correlated features, an improved VFPU algorithm is employed for vertical federated semi-supervised learning, which iteratively generates and filters pseudo-labels, effectively leveraging unlabeled data across different tasks (classification/regression); third, for features with low correlations, generative models such as TabDDPM or VF-GAIN are introduced for data synthesis. Experiments conducted on Bank, Credit, Letter, and News datasets demonstrate that optimal performance is achieved with a correlation threshold (τ) of 0.6, a confidence threshold (α) of 0.7, and the use of GBDT as the base learner. Moreover, this method maintains stable superiority even with a sample missing rate as high as 80\%.
\\


\noindent\textbf{Keywords:} 
\begin{minipage}[t]{0.85\linewidth}
	Federated Learning; Semi-Supervised Learning; Sample Generation; Privacy Protection; Vertical Federated Learning
\end{minipage}

\clearpage