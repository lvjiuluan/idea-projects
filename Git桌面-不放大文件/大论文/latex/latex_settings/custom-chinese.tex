%设置文章格式


\usepackage[final]{pdfpages}

\usepackage{geometry}
\usepackage{makecell}
\usepackage{caption}
\usepackage{multirow}
\usepackage{booktabs}
\makeatletter
\let\c@lofdepth\relax
\let\c@lotdepth\relax
\makeatother
\usepackage{subfigure}
\usepackage{float}
\usepackage[titles,subfigure]{tocloft}
\usepackage{soul}
\usepackage{color,xcolor}
\usepackage{setspace}
\usepackage{titletoc}
\usepackage{longtable}

\usepackage[list=off]{bicaption}
\usepackage[hidelinks]{hyperref}
%\usepackage{mathptmx} %设置数学公式为新罗马字体
\usepackage{amsmath} %几个数学宏包
\usepackage{fontspec}
%\usepackage{mathspec}
%\usepackage[no-math]{fontspec}
\usepackage{newtxmath}
\usepackage{bm}
%\usepackage{newtxtext}
%\usepackage{unicode-math}
%\usepackage{txfonts}
\usepackage{diagbox}


\AtBeginDocument{\DeclareMathAlphabet{\mathbf}{OT1}{cmr}{bx}{n}}







% 设置页边距
\geometry{a4paper,top=3cm,bottom=3cm,left=3cm,right=3cm}

% 设置行间距 1.5倍
%\linespread{1.4}\selectfont

% 设置段与段之间的垂直距离 \parskip默认橡皮长度是0pt plus 1pt
\setlength{\parskip}{0pt}
%\setlength{\baselineskip}{20pt}
% \setlength{\parindent}{0pt}

% 行间距={}*字体里面的第二个{},对于xiaosi而言就是1等于20磅
\linespread{1}\selectfont

%设置字体
% 设置英文字体
\setmainfont{Times New Roman}[
    BoldFont = Times New Roman Bold,
    ItalicFont = Times New Roman Italic,
    BoldItalicFont = Times New Roman Bold Italic
]

\setsansfont{Times New Roman}[
    BoldFont = Times New Roman Bold,
    ItalicFont = Times New Roman Italic,
    BoldItalicFont = Times New Roman Bold Italic
]

\setmonofont{Times New Roman}[
    BoldFont = Times New Roman Bold,
    ItalicFont = Times New Roman Italic,
    BoldItalicFont = Times New Roman Bold Italic
]

%设置字号,举例\xiaosi表示的是20磅行距,\xiaosid对应的是单倍行距


\usepackage{ctexsize,type1cm}
\newcommand{\chuhaod}{\fontsize{42pt}{54.6pt}\selectfont}
\newcommand{\xiaochud}{\fontsize{36pt}{46.8pt}\selectfont}
%\newcommand{\yihao}{\fontsize{26pt}{39pt}\selectfont}
%\newcommand{\xiaoyi}{\fontsize{24pt}{36pt}\selectfont}   
\newcommand{\erhaod}{\fontsize{22pt}{28.6pt}\selectfont}          
\newcommand{\xiaoerd}{\fontsize{18pt}{23.4pt}\selectfont}          
\newcommand{\sanhaod}{\fontsize{16pt}{20.8pt}\selectfont}        
\newcommand{\xiaosand}{\fontsize{15pt}{19.5pt}\selectfont}        
\newcommand{\sihaod}{\fontsize{14pt}{18.2pt}\selectfont}            
\newcommand{\xiaosid}{\fontsize{12pt}{15.6pt}\selectfont}            
\newcommand{\wuhaod}{\fontsize{10.5pt}{13.65pt}\selectfont}
%\newcommand{\xiaowu}{\fontsize{9pt}{13.5pt}\selectfont}    
%\newcommand{\liuhao}{\fontsize{7.5pt}{11.25pt}\selectfont}

\newcommand{\wuhaob}{\fontsize{10.5pt}{17pt}\selectfont}

\newcommand{\chuhao}{\fontsize{42pt}{42pt}\selectfont}
\newcommand{\xiaochu}{\fontsize{36pt}{36pt}\selectfont}
\newcommand{\yihao}{\fontsize{26pt}{39pt}\selectfont}
\newcommand{\xiaoyi}{\fontsize{24pt}{36pt}\selectfont}   
%\newcommand{\erhao}{\fontsize{22pt}{33pt}\selectfont}          
\newcommand{\xiaoer}{\fontsize{18pt}{27pt}\selectfont}          
\newcommand{\sanhao}{\fontsize{16pt}{20pt}\selectfont}        
\newcommand{\xiaosan}{\fontsize{15pt}{20pt}\selectfont}        
\newcommand{\sihao}{\fontsize{14pt}{20pt}\selectfont}            
\newcommand{\xiaosi}{\fontsize{12pt}{20pt}\selectfont}            
\newcommand{\wuhao}{\fontsize{10.5pt}{20pt}\selectfont}
\newcommand{\xiaowu}{\fontsize{9pt}{13.5pt}\selectfont}    
\newcommand{\liuhao}{\fontsize{7.5pt}{11.25pt}\selectfont}

%使用公式,表格,图片
% 先取消原先的 \Bbbk 宏的定义,避免与后续宏包加载产生冲突。
\let\Bbbk\relax
\usepackage{mathtools,amsmath,amssymb,graphicx,array,float}


%按照章节编号
\numberwithin{figure}{chapter}
\numberwithin{table}{chapter}
\numberwithin{equation}{chapter}

%图、表、公式格式改为 X-X
\renewcommand\thefigure{\arabic{chapter}-\arabic{figure}}
\captionsetup[figure]{labelsep=space}

\renewcommand\thetable{\arabic{chapter}-\arabic{table}}
\captionsetup[table]{labelsep=space}

\renewcommand\theequation{\arabic{chapter}-\arabic{equation}}

%设置图表英文标题格式
\captionsetup{font={stretch=1.2}} %调整中英文图表标题的行距
\setlength{\abovecaptionskip}{6pt} %调整图表标题与图片和表格之间的距离
%\setlength{\belowcaptionskip}{1pt} 
\captionsetup[figure][bi-second]{name=\wuhao Fig.}
\captionsetup[table][bi-second]{name=\wuhao Table}
%\setlength{\captionwidth}{10cm} 




% 设置页眉面脚
%% 设置章节前的页码格式
%\usepackage[pagestyles]{titlesec}
%\newpagestyle{MyStyle}{
%  \setfoot{}{\Roman{page}}{}
%%  \headrule
%}

\usepackage{fancyhdr}

%\fancypagestyle{abstract}
%{
%	\fancyhf{}
%	\renewcommand{\headrulewidth}{0.5pt}
%	%	\renewcommand{\footrulewidth}{0mm}
%	\fancyfoot[C]{\songti\xiaowu \Roman{page}}
%	\fancyhead[C]{\wuhao \leftmark}
%}

%重新设置plain,chapter设置页眉时会调用plain,因此需要重新定义plain,不能设置为其他名称



\fancypagestyle{plain}{
	\fancyhf{}
	\fancyfoot[C]{\songti\xiaowu  \Roman{page} }
	\fancyhead[C]{\songti\wuhao \leftmark}
}





%页眉设置,博士和硕士学位论文请在下面自行修改
\fancypagestyle{body}{
    \fancyhf{}
    \fancyfoot[C]{\songti\xiaowu \thepage}
    \fancyhead[CO]{\songti\wuhao \rightmark}
    \fancyhead[CE]{\songti\wuhao 重庆邮电大学博士学位论文}
}

\fancypagestyle{others}{
	\fancyhf{}
	\fancyfoot[C]{\songti\xiaowu \thepage}
	\fancyhead[CO]{\songti\wuhao \leftmark}
	\fancyhead[CE]{\songti\wuhao 重庆邮电大学博士学位论文}
}







%设置双线页眉
%\makeatletter
%\def\headrule{
%    {\if@fancyplain\let\headrulewidth\plainheadrulewidth\fi%
%    \hrule\@height 1.0pt \@width\headwidth\vskip1pt%上面线为1pt粗
%    \hrule\@height 0pt \@width\headwidth  %下面0.5pt粗
%    \vskip-2\headrulewidth\vskip-1.2pt}    %两条线的距离1pt
%    \vspace{6mm}}     %双线与下面正文之间的垂直间距
%\makeatother

%设置双线页脚
\makeatletter
\def\footrule{
    {\if@fancyplain\let\footrulewidth\plainfootrulewidth\fi%
    \hrule\@height 0pt \@width\headwidth          %上面0.5pt粗
    \vskip 1pt
    \hrule\@height 0pt \@width\headwidth %下面线为1pt粗
    \vskip-2\headrulewidth\vskip-1.2pt}    %两条线的距离1pt
    \vspace{8mm}}     %双线与下面正文之间的垂直间距
\makeatother

%\renewcommand\thechapter{\arabic{chapter}}

%设置文章格式
\ctexset {
    contentsname={目 \quad 录},
    listfigurename={图目录},
    listtablename={表目录},
    figurename={\wuhao 图},
    tablename={\wuhao 表},
    bibname={参考文献},
    appendixname={附录},
    chapter={
    	name={第,章},
    	aftername=\enspace,
    	number={\arabic{chapter}},
        beforeskip={-2pt},
        afterskip={18pt},
        nameformat={\heiti\sanhao\centering\mdseries}, 
        titleformat={\heiti\sanhao\centering\mdseries},
    },
    section={
    	aftername=\enspace,
    	beforeskip={18pt},
    	afterskip={6pt},
        format={\heiti\sihao\leftline},
    },
    subsection={
    	aftername=\enspace,
    	beforeskip={12pt},
    	afterskip={6pt},
        format={\heiti\sihao\leftline},
    },
    subsubsection={
    	aftername=\enspace,
    	beforeskip={12pt},
    	afterskip={6pt},
        format={\heiti\xiaosi\leftline},
    }
}



% 目录中的章加点
%\usepackage[titles]{tocloft}
%\renewcommand{\cftdot}{$\cdot$}
%\renewcommand{\cftdotsep}{1.5}
%\setlength{\cftbeforechapskip}{10pt}
%
%\renewcommand{\cftchapleader}{\cftdotfill{\cftchapdotsep}}
%\renewcommand{\cftchapdotsep}{\cftdotsep}
%\makeatletter
%\renewcommand{\numberline}[1]{
%\settowidth\@tempdimb{#1\hspace{0.5em}}
%\ifdim\@tempdima<\@tempdimb
%  \@tempdima=\@tempdimb
%\fi
%\hb@xt@\@tempdima{\@cftbsnum #1\@cftasnum\hfil}\@cftasnumb}
%\makeatother

% 设置目录字体尺寸
%\renewcommand{\cftchapfont}{\heiti\xiaosi}
%\renewcommand{\cftsecfont}{\heiti\xiaosi}
%\renewcommand{\cftsubsecfont}{\songti\xiaosi}
%\renewcommand{\cftsubsubsecfont}{\songti\xiaosi}

% 设置目录标题深度
\setcounter{secnumdepth}{3}
\setcounter{tocdepth}{2}


% 设置目录标题缩进,字体等
\titlecontents{chapter}[3.8em]{\heiti\xiaosi}{\contentslabel{3.8em}}{\hspace{-3.76em}}{\titlerule*[0.4pc]{$\cdot$}\contentspage\hspace*{0.01em}}
\titlecontents{section}[3.8em]{\songti\xiaosi}{\contentslabel{2em}}{\hspace{-2em}}{\titlerule*[0.4pc]{$\cdot$}\contentspage\hspace*{0.01em}}
\titlecontents{subsection}[6.5em]{\songti\xiaosi}{\contentslabel{2.7em}}{\hspace{-2.7em}}{\titlerule*[0.4pc]{$\cdot$}\contentspage\hspace*{0.01em}}

%设置图表目录标题格式
\titlecontents{figure}[0pt]{\songti\xiaosi\settowidth{\hangindent}{图~\thecontentslabel\ \ }}{图~\thecontentslabel\ \ }{}{\hspace{.25em}\titlerule*[4pt]{$\cdot$}\contentspage}


\titlecontents{table}[0pt]{\songti\xiaosi\settowidth{\hangindent}{表~\thecontentslabel\ \ }}{表~\thecontentslabel\ \ }{}{\hspace{.25em}\titlerule*[4pt]{$\cdot$}\contentspage}

%\dottedcontents{subsubsection}[4cm]{\normalsize}{4em}{4pt}

%使用代码排版包
\usepackage{listings}
\usepackage{color}
\lstset{%
    frame=shadowbox,
    extendedchars=false,            % 不使用xelatex而使用CJK方式处理汉字
    language=python,
    basicstyle=\sffamily,           % 设置整体格式
    keywordstyle=\bfseries,         % 关键字格式
    commentstyle=\rmfamily\itshape, % 注释格式
    stringstyle=\ttfamily,          % 字符串格式
    columns=flexible,
    escapechar=',                   % 注释中显示汉字,eg //'一个整数'
    tabsize=4,
    numbers=left,
    numberstyle=\small,             % 行号字体设置
    stepnumber=1,                   % 行号距离设置,1代表每行加行号
    numbersep=8pt,                  % 行号和代码距离设置
    backgroundcolor=\color{white},
    showspaces=false,               % show spaces adding particular underscores
    showstringspaces=false,         % 使用下划线连接字符串
    showtabs=false,
    frame=single,                   % 给代码加边框
    captionpos=b,                   % sets the caption-position to bottom
    breaklines=true,                % 自动换行设置
    breakatwhitespace=false,        % sets if automatic breaks should only happen at whitespace
    escapeinside={\%*}{*)},         % if you want to add a comment within your code
    xleftmargin=2em,                % 设置左边距,宽度默认是与页芯等宽的
    xrightmargin=2em,               % 设置右边距,宽度默认是与页芯等宽的
    aboveskip=1em                   % 设置上边距
}

%设置自定义变量
\newcommand\degree{^\cire}

% 定义文献引用格式,\cite正常引用 \supercite右上角引用
\usepackage{cite}
\newcommand{\upcite}[1]{\textsuperscript{\textsuperscript{\cite{#1}}}}
\newcommand\supercite[2][]{%
\textsuperscript{\cite[#1]{#2}}
}

\usepackage{enumitem}
\setlist[description]{
    itemsep=-5pt,
    font=\songti,
}

% 定义中文封面环境
\newenvironment{titletabbing}
{\par\bfseries\songti\sihao\tabbing}
{\endtabbing\par}

\usepackage[nottoc]{tocbibind}
\endinput
