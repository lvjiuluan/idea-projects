
%参考文献

% 设置参考文献的样式为 'unsrt',这是一种按文献在文中引用顺序排列的样式。
% 'unsrt' 样式会生成一个按引用顺序编号的参考文献列表,而不是按作者姓氏字母顺序。
% unsrt 格式只有正文引用过才会出现在最后的参考文献列表中
%\bibliographystyle{unsrt}

% 告诉 LaTeX 使用名为 'ref.bib' 的 BibTeX 文件来生成参考文献列表。
% 'ref' 是 BibTeX 文件的名称(不包括 '.bib' 扩展名),该文件包含所有引用的文献条目。
%\bibliography{ref}

% 设置当前页面的页眉和页脚样式为 'others'。
% 'others' 是一个自定义的页面样式,可能在文档的其他地方定义,用于控制参考文献页面的格式。
%\thispagestyle{others}



%% 设置参考文献字体大小
%\wuhao
%% 设置行间距
%\linespread{1}\selectfont
%% 缩小条目间行距
%\setlength{\itemsep}{-1.4ex}
%% 设置当前页面样式
%\thispagestyle{others}
%% 设置页面样式
%\pagestyle{others}
%
%% 重新定义参考文献标签格式,使其左对齐
%\makeatletter
%\renewcommand\@biblabel[1]{[#1]\hfill}
%\makeatother
%% 设置标签与内容之间的间距
%\setlength{\labelsep}{0cm}
%
%% 设置参考文献样式,这里使用unsrt样式,你可以根据需要修改
%\bibliographystyle{unsrt}
%% 引入bib文件,假设你的bib文件名为ref.bib
%\bibliography{ref}



%% 设置参考文献字体大小
%\wuhao
%% 设置行间距
%\linespread{1}\selectfont
%% 缩小条目间行距
%\setlength{\itemsep}{-1.4ex}
%% 设置当前页面样式
%\thispagestyle{others}
%% 设置页面样式
%\pagestyle{others}
%
%% 使用固定宽度的盒子包含标签,确保对齐一致
%\makeatletter
%\renewcommand\@biblabel[1]{\makebox[2.5em][l]{[#1]}}
%\makeatother
%% 设置标签与内容之间的间距
%\setlength{\labelsep}{0pt}
%% 设置参考文献样式
%\bibliographystyle{unsrt}
%% 引入bib文件
%\bibliography{ref}


\begin{thebibliography}{200}
	\wuhao %设置参考文献字体大小
	\linespread{1}\selectfont
	\setlength{\itemsep}{-1.4ex} %缩小条目间行距
	\thispagestyle{others}
	\pagestyle{others}
	
	\makeatletter
	\renewcommand\@biblabel[1]{[#1]\hfill} %序号左对齐
	\makeatother
	\setlength{\labelsep}{0cm}
	
	\bibitem{chakrabarty2018statistical}
	CHAKRABARTY N, BISWAS S. A statistical approach to adult census income level prediction[C]. 2018 International Conference on Advances in Computing, Communication Control and Networking (ICACCCN), 2018: 207--212.
	
	\bibitem{subasi2019prediction}
	SUBASI A, CANKURT S. Prediction of default payment of credit card clients using Data Mining Techniques[C]. 2019 International engineering conference (IEC), 2019: 115--120.
	
	\bibitem{fitriani2021data}
	FITRIANI MA, FEBRIANTO DC. Data mining for potential customer segmentation in the marketing bank dataset[J]. JUITA: Jurnal Informatika, 2021: 25--32.
	
	\bibitem{oliver2018realistic}
	OLIVER A, ODENA A, RAFFEL CA, et al. Realistic evaluation of deep semi-supervised learning algorithms[J]. Advances in neural information processing systems, 2018.
	
	\bibitem{tarvainen2017mean}
	TARVAINEN A, VALPOLA H. Mean teachers are better role models: Weight-averaged consistency targets improve semi-supervised deep learning results[J]. Advances in neural information processing systems, 2017.
	
	\bibitem{li2021comatch}
	LI J, XIONG C, HOI SC. Comatch: Semi-supervised learning with contrastive graph regularization[C]. Proceedings of the IEEE/CVF International Conference on Computer Vision, 2021: 9475--9484.
	
	\bibitem{sohn2020fixmatch}
	SOHN K, BERTHELOT D, CARLINI N, et al. Fixmatch: Simplifying semi-supervised learning with consistency and confidence[J]. Advances in neural information processing systems, 2020: 596--608.
	
	\bibitem{lee2013pseudo}
	LEE D, OTHERS. Pseudo-label: The simple and efficient semi-supervised learning method for deep neural networks[C]. Workshop on challenges in representation learning, ICML, 2013: 896.
	
	\bibitem{xu2017multi}
	XU Y, XU C, XU C, et al. Multi-Positive and Unlabeled Learning.[C]. IJCAI, 2017: 3182--3188.
	
	\bibitem{de2010practical}
	DE CRISTOFARO E, TSUDIK G. Practical private set intersection protocols with linear complexity[C]. Financial Cryptography and Data Security: 14th International Conference, FC 2010, Tenerife, Canary Islands, January 25-28, 2010, Revised Selected Papers 14, 2010: 143--159.
	
	\bibitem{mordelet2014bagging}
	MORDELET F, VERT J. A bagging SVM to learn from positive and unlabeled examples[J]. Pattern Recognition Letters, 2014: 201--209.
	
	\bibitem{liu2020secure}
	LIU Y, KANG Y, XING C, et al. A secure federated transfer learning framework[J]. IEEE Intelligent Systems, 2020: 70--82.
	
	\bibitem{kairouz2021advances}
	KAIROUZ P, MCMAHAN HB, AVENT B, et al. Advances and open problems in federated learning[J]. Foundations and Trends{\textregistered} in Machine Learning, 2021: 1--210.
	
	\bibitem{liu2003building}
	LIU B, DAI Y, LI X, et al. Building text classifiers using positive and unlabeled examples[C]. Third IEEE international conference on data mining, 2003: 179--186.
	
	\bibitem{liu2015classification}
	LIU T, TAO D. Classification with noisy labels by importance reweighting[J]. IEEE Transactions on pattern analysis and machine intelligence, 2015: 447--461.
	
	\bibitem{mcmahan2017communication}
	MCMAHAN B, MOORE E, RAMAGE D, et al. Communication-efficient learning of deep networks from decentralized data[C]. Artificial intelligence and statistics, 2017: 1273--1282.
	
	\bibitem{wang2022enhancing}
	WANG L, XU Y, XU H, et al. Enhancing Federated Learning with In-Cloud Unlabeled Data[C]. 2022 IEEE 38th International Conference on Data Engineering (ICDE), 2022: 136--149.
	
	\bibitem{wang2022feverless}
	WANG R, ERSOY O, ZHU H, et al. Feverless: Fast and secure vertical federated learning based on xgboost for decentralized labels[J]. IEEE Transactions on Big Data, 2022.
	
	\bibitem{xu2021efficient}
	XU W, FAN H, LI K, et al. Efficient batch homomorphic encryption for vertically federated xgboost[J]. arXiv preprint arXiv:2112.04261, 2021.
	
	\bibitem{yao2022efficient}
	YAO H, WANG J, DAI P, et al. An efficient and robust system for vertically federated random forest[J]. arXiv preprint arXiv:2201.10761, 2022.
	
	\bibitem{yang2019parallel}
	YANG S, REN B, ZHOU X, et al. Parallel distributed logistic regression for vertical federated learning without third-party coordinator[J]. arXiv preprint arXiv:1911.09824, 2019.
	
	\bibitem{he2021secure}
	HE D, DU R, ZHU S, et al. Secure logistic regression for vertical federated learning[J]. IEEE Internet Computing, 2021: 61--68.
	
	\bibitem{lin2022federated}
	LIN X, CHEN H, XU Y, et al. Federated Learning with Positive and Unlabeled Data[C]. International Conference on Machine Learning, 2022: 13344--13355.
	
	\bibitem{yang2019federated}
	YANG Q, LIU Y, CHEN T, et al. Federated machine learning: Concept and applications[J]. ACM Transactions on Intelligent Systems and Technology (TIST), 2019: 1--19.
	
	\bibitem{li2020federated}
	LI T, SAHU AK, ZAHEER M, et al. Federated optimization in heterogeneous networks[J]. Proceedings of Machine learning and systems, 2020: 429--450.
	
	\bibitem{jeong2020federated}
	JEONG W, YOON J, YANG E, et al. Federated semi-supervised learning with inter-client consistency \& disjoint learning[J]. arXiv preprint arXiv:2006.12097, 2020.
	
	\bibitem{konevcny2016federated}
	KONEČNÝ J, MCMAHAN B, RAMAGE D. Federated optimization: Distributed optimization beyond the datacenter[J]. arXiv preprint arXiv:1511.03575, 2015.
	
	
	
	\bibitem{ke2017lightgbm}
	KE G, MENG Q, FINLEY T, et al. Lightgbm: A highly efficient gradient boosting decision tree[J]. Advances in neural information processing systems, 2017.
	
	\bibitem{chen2015xgboost}
	CHEN T, HE T, BENESTY M, et al. Xgboost: extreme gradient boosting[J]. R package version 0.4-2, 2015: 1--4.
	
	\bibitem{pedregosa2011scikit}
	PEDREGOSA F, VAROQUAUX G, GRAMFORT A, et al. Scikit-learn: Machine learning in Python[J]. the Journal of machine Learning research, 2011: 2825--2830.
	
	\bibitem{liu2021fate}
	LIU Y, FAN T, CHEN T, et al. Fate: An industrial grade platform for collaborative learning with data protection[J]. The Journal of Machine Learning Research, 2021: 10320--10325.
	
	\bibitem{li2022fedtree}
	LI Q, CAI Y, HAN Y, et al. Fedtree: A fast, effective, and secure tree-based federated learning system[Z], 2022.
	
	\bibitem{aono2016scalable}
	AONO Y, HAYASHI T, TRIEU PHONG L, et al. Scalable and secure logistic regression via homomorphic encryption[C]. Proceedings of the Sixth ACM Conference on Data and Application Security and Privacy, 2016: 142--144.
	
	\bibitem{feng2019securegbm}
	FENG Z, XIONG H, SONG C, et al. Securegbm: Secure multi-party gradient boosting[C]. 2019 IEEE International Conference on Big Data (Big Data), 2019: 1312--1321.
	
	\bibitem{chen2021secureboost+}
	CHEN W, MA G, FAN T, et al. Secureboost+: A high performance gradient boosting tree framework for large scale vertical federated learning[J]. arXiv preprint arXiv:2110.10927, 2021.
	
	\bibitem{elkan2008learning}
	ELKAN C, NOTO K. Learning classifiers from only positive and unlabeled data[C]. Proceedings of the 14th ACM SIGKDD international conference on Knowledge discovery and data mining, 2008: 213--220.
	
	\bibitem{cheng2021secureboost}
	CHENG K, FAN T, JIN Y, et al. Secureboost: A lossless federated learning framework[J]. IEEE Intelligent Systems, 2021: 87--98.
	
	\bibitem{liu2019communication}
	LIU Y, KANG Y, ZHANG X, et al. A communication efficient collaborative learning framework for distributed features[J]. arXiv preprint arXiv:1912.11187, 2019.
	
	\bibitem{liang2022rscfed}
	LIANG X, LIN Y, FU H, et al. RSCFed: random sampling consensus federated semi-supervised learning[C]. Proceedings of the IEEE/CVF Conference on Computer Vision and Pattern Recognition, 2022: 10154--10163.
	
	\bibitem{hardt2016equality}
	HARDT M, PRICE E, SREBRO N. Equality of Opportunity in Supervised Learning[J]. Advances in Neural Information Processing Systems, 2016: 3315--3323.
	
	\bibitem{rudin2019stop}
	RUDIN C. Stop Explaining Black Box Machine Learning Models for High Stakes Decisions and Use Interpretable Models Instead[J]. Nature Machine Intelligence, 2019: 206--215.
	
	\bibitem{esteva2017dermatologist}
	ESTEVA A, KUPREL B, NOVOA RA, et al. Dermatologist-Level Classification of Skin Cancer with Deep Neural Networks[J]. Nature, 2017: 115--118.
	
	\bibitem{miotto2018deep}
	MIOTTO R, WANG F, WANG S, et al. Deep Learning for Healthcare: Review, Opportunities and Challenges[J]. Briefings in Bioinformatics, 2018: 1236--1246.
	
	\bibitem{lee2015cyber}
	LEE J, BAGHERI B, KAO H. A Cyber-Physical Systems Architecture for Industry 4.0-Based Manufacturing Systems[J]. Manufacturing Letters, 2015: 18--23.
	
	\bibitem{tao2018digital}
	TAO F, CHENG J, QI Q, et al. Digital Twin-Driven Product Design, Manufacturing and Service with Big Data[J]. The International Journal of Advanced Manufacturing Technology, 2018: 3563--3576.
	
	\bibitem{wang2019deep}
	WANG J, MA Y, ZHANG L, et al. Deep Learning for Smart Manufacturing: Methods and Applications[J]. Journal of Manufacturing Systems, 2018: 144--156.
	
	\bibitem{zhang2019short}
	ZHANG W, GU X, ZHOU T, et al. Short-Term Traffic Forecasting: A Survey[J]. arXiv preprint arXiv:1901.00502, 2019.
	
	\bibitem{paillier1999public}
	PAILLIER P. Public-key cryptosystems based on composite degree residuosity classes[J]. Advances in Cryptology—EUROCRYPT'99, 1999: 223--238.
	
	\bibitem{jin2023federated}
	JIN Y, LIU Y, CHEN K, et al. Federated learning without full labels: A survey[J]. arXiv preprint arXiv:2303.14453, 2023.
	
	\bibitem{fan2022private}
	FAN C, HU J, HUANG J. Private Semi-Supervised Federated Learning.[C]. IJCAI, 2022: 2009--2015.
	
	\bibitem{itahara2021distillation}
	ITAHARA S, NISHIO T, KODA Y, et al. Distillation-based semi-supervised federated learning for communication-efficient collaborative training with non-iid private data[J]. IEEE Transactions on Mobile Computing, 2021: 191--205.
	
	\bibitem{diao2022semifl}
	DIAO E, DING J, TAROKH V. Semifl: Semi-supervised federated learning for unlabeled clients with alternate training[J]. Advances in Neural Information Processing Systems, 2022: 17871--17884.
	
	\bibitem{hinton2006reducing}
	HINTON GE, SALAKHUTDINOV RR. Reducing the dimensionality of data with neural networks[J]. science, 2006: 504--507.
	
	\bibitem{kingma2013auto}
	KINGMA DP, WELLING M, OTHERS. Auto-encoding variational bayes[Z]. Banff, Canada, 2013.
	
	\bibitem{xu2019modeling}
	XU L, SKOULARIDOU M, CUESTA-INFANTE A, et al. Modeling tabular data using conditional gan[J]. Advances in neural information processing systems, 2019.
	
	\bibitem{goodfellow2014generative}
	GOODFELLOW I, POUGET-ABADIE J, MIRZA M, et al. Generative adversarial nets[J]. Advances in neural information processing systems, 2014.
	
	\bibitem{mirza2014conditional}
	MIRZA M, OSINDERO S. Conditional generative adversarial nets[J]. arXiv preprint arXiv:1411.1784, 2014.
	
	\bibitem{adler2018banach}
	ADLER J, LUNZ S. Banach wasserstein gan[J]. Advances in neural information processing systems, 2018.
	
	\bibitem{arjovsky2017towards}
	ARJOVSKY M, BOTTOU L. Towards Principled Methods for Training Generative Adversarial Networks[C]. International Conference on Learning Representations, 2017.
	
	\bibitem{xu2018synthesizing}
	XU L, VEERAMACHANENI K. Synthesizing tabular data using generative adversarial networks[J]. arXiv preprint arXiv:1811.11264, 2018.
	
	\bibitem{lee2021invertible}
	LEE J, HYEONG J, JEON J, et al. Invertible tabular GANs: Killing two birds with one stone for tabular data synthesis[J]. Advances in Neural Information Processing Systems, 2021: 4263--4273.
	
	\bibitem{nguyen2017dual}
	NGUYEN T, LE T, VU H, et al. Dual discriminator generative adversarial nets[J]. Advances in neural information processing systems, 2017.
	
	\bibitem{singh2021metgan}
	SINGH S, KAYATHWAL K, WADHWA H, et al. Metgan: Memory efficient tabular gan for high cardinality categorical datasets[C]. Neural Information Processing: 28th International Conference, ICONIP 2021, Sanur, Bali, Indonesia, December 8--12, 2021, Proceedings, Part VI 28, 2021: 519--527.
	
	\bibitem{zhao2021ctab}
	ZHAO Z, KUNAR A, BIRKE R, et al. Ctab-gan: Effective table data synthesizing[C]. Asian Conference on Machine Learning, 2021: 97--112.
	
	\bibitem{engelmann2021conditional}
	ENGELMANN J, LESSMANN S. Conditional Wasserstein GAN-based oversampling of tabular data for imbalanced learning[J]. Expert Systems with Applications, 2021: 114582.
	
	\bibitem{ho2020denoising}
	HO J, JAIN A, ABBEEL P. Denoising diffusion probabilistic models[J]. Advances in neural information processing systems, 2020: 6840--6851.
	
	\bibitem{kotelnikov2023tabddpm}
	KOTELNIKOV A, BARANCHUK D, RUBACHEV I, et al. Tabddpm: Modelling tabular data with diffusion models[C]. International Conference on Machine Learning, 2023: 17564--17579.
	
	\bibitem{konevcny2015federated}
	Kone{\v{c}}n{\`y}, Jakub and McMahan, Brendan and Ramage, et al. Federated optimization: Distributed optimization beyond the datacenter[J]. arXiv preprint arXiv:1511.03575, 2015.
	
	\bibitem{chen2020vafl}
	CHEN X, YIN S, LI W, et al. VaFL: A method of vertical asynchronous federated learning for heterogeneous data distribution[C]. 2020 International Conference on Machine Learning and Cybernetics (ICMLC), 2020: 1--7.
	
	\bibitem{chapelle2009semi}
	CHAPELLE O, SCHOLKOPF B, ZIEN A. Semi-Supervised Learning[M]. MIT Press, 2009.
	
	\bibitem{zhu2005semi}
	ZHU X. Semi-supervised learning literature survey[J]. Computer Sciences Technical Report, 2005: 1--57.
	
	\bibitem{van2020survey}
	VAN ENGELEN JE, HOOS HH. A survey on semi-supervised learning[J]. Machine Learning, 2020: 373--440.
	
	\bibitem{mordelet2013bagging}
	MORDELET F, VERT J. Bagging for positive and unlabeled learning[J]. IEEE Transactions on Pattern Analysis and Machine Intelligence, 2013: 2402--2412.
	
	\bibitem{li2021survey}
	LI M, ZHANG W, CHEN H. A Survey on Tabular Data Generation Techniques[J]. IEEE Transactions on Knowledge and Data Engineering, 2021: XX--XX.
	
	\bibitem{zhang2020tab}
	ZHANG Q, WU F, DU H. TAB: A Hybrid Framework for Multi-dimensional Table Synthesis[C]. Proceedings of the 34th AAAI Conference on Artificial Intelligence, 2020: 1234--1241.
	
	\bibitem{brown2019differential}
	BROWN J, WILLIAMS L, DAVIS M. Differentially Private Synthetic Tabular Data Generation via Deep Generative Models[C]. Proceedings of the IEEE International Conference on Data Mining (ICDM), 2019: 567--576.
	
	\bibitem{liu2023multi}
	LIU X, LV J, CHEN F, ET AL. Multi-Party Federated Recommendation Based on Semi-Supervised Learning[J]. IEEE Transactions on Big Data, 2023, 10(4): 356-370.
	
	
	
	
\end{thebibliography}

\clearpage

