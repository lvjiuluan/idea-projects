%中文摘要,自行编辑内容



\chapter{摘\quad 要}
\xiaosi

随着数据隐私保护需求的日益增加以及分布式数据环境的广泛存在,联邦学习(Federated Learning, FL)作为一种保护数据隐私的分布式机器学习范式得到了快速发展。然而,在实际应用中,联邦学习面临标记数据稀缺、数据分布异构以及参与方样本对齐不足等挑战。半监督学习(Semi-Supervised Learning, SSL)因其能够有效利用少量标记数据和大量未标记数据提升模型性能,成为解决上述问题的重要手段。本文聚焦于基于联邦半监督学习的样本生成方法研究,旨在通过设计创新的算法和框架,在保护数据隐私的前提下,充分利用未标记数据和非对齐样本,提升联邦学习模型的泛化能力和应用价值。  

本文首先针对多方联邦学习中未标记数据缺失的问题,提出了一种基于多方联邦的半监督学习方法VFPU。该方法通过分析未标记数据缺失问题(UDD-PU)的特性,设计了结合正样本-未标记样本(Positive-Unlabeled, PU)学习技术的联邦学习框架,利用多方协作的未标记数据进行模型训练。VFPU采用加密样本对齐、动态伪标签生成和集成学习策略,在保护数据隐私的同时,显著提升了推荐任务的精确率和召回率。实验表明,在Bank、Credit和Census数据集上,VFPU的AUC值分别达到0.886、0.639和0.854,相较于传统集中式PU学习方法,性能下降仅为1.5\%至3.2\%,验证了隐私保护与模型性能的有效平衡。  

其次,针对纵向联邦学习(Vertical Federated Learning, VFL)中对齐样本数量有限、非对齐样本未被充分利用的局限性,本文提出了基于半监督学习的样本生成方法VFPU-M-Syn。该方法通过跨方特征相关性分析、半监督预测和生成模型合成三种策略,生成高质量的伪标签和合成数据。具体而言,VFPU-M-Syn计算跨方特征相关性以识别参与方数据间的依赖关系,利用纵向联邦半监督学习预测未标记样本的伪标签,并针对低相关性特征引入生成对抗网络(GAN)合成数据。 

本文的研究成果不仅丰富了联邦半监督学习领域的理论体系,还为医疗、金融、智能制造等需要隐私保护的分布式学习场景提供了可行的解决方案。未来工作将进一步优化算法在非独立同分布(Non-IID)数据下的适应性,探索动态阈值调整机制,并拓展其在跨模态数据融合中的应用,为数据要素的高效利用与安全共享提供技术支撑。
  
\noindent\songti\textbf{关键词:}联邦学习;半监督学习;样本生成;隐私保护;纵向联邦学习

\clearpage